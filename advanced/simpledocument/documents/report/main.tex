% ========================================
% FileName: main.tex
% Date: 21 juin 2023 - 13:16
% Author: Ammar Mian
% Email: ammar.mian@univ-smb.fr
% GitHub: https://github.com/ammarmian
% Brief: Example document of a report
%        on IRIS dataset using LaTeX
% =========================================

\documentclass[letterpaper]{ieeeconf}

\usepackage[utf8]{inputenc}
\usepackage{tikz}
\usepackage{pgfplots}
\usepackage{amsmath}
\usepackage{kantlipsum}

\title{Example report on IRIS dataset}
\author{Ammar Mian}


\begin{document}

\maketitle

Some blabla about the example.

\kant[1]

\begin{figure}[h]
    \centering
    \input{exports/summary_statistics/histogram_petal_width_(cm).tex}
    \caption{Histogram of petal width}
    \label{fig:histogram_petal_width}
\end{figure}

\kant[2]

\begin{figure}[h]
    \centering
    \input{exports/summary_statistics/histogram_petal_length_(cm).tex}
    \caption{Histogram of petal length}
    \label{fig:histogram_petal_length}
\end{figure}

\begin{figure}[h]
    \centering
    \input{exports/summary_statistics/histogram_sepal_width_(cm).tex}
    \caption{Histogram of sepal width}
    \label{fig:histogram_sepal_width}
\end{figure}

\kant[3]

\begin{figure}[h]
    \centering
    \input{exports/summary_statistics/histogram_sepal_length_(cm).tex}
    \caption{Histogram of sepal length}
    \label{fig:histogram_sepal_length}
\end{figure}

\kant[4]

\end{document}


